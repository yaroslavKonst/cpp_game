\documentclass[12pt]{article}
\usepackage[russian]{babel}
\usepackage{amsmath}
\usepackage{fontspec}
\usepackage[T1,T2A]{fontenc}
\usepackage{graphicx}
\usepackage{listings}
\setmainfont{Liberation Serif}
\oddsidemargin = -5mm
\topmargin = -20mm
\textheight = 230mm
\textwidth = 170mm
\usepackage{indentfirst}
\usepackage{xcolor}


\begin{document}

\renewcommand{\contentsname}{Содержание}

\newcommand{\newsection}[1]
{
	\suppressfloats[t]

	\section{#1}
}

\newcommand{\newsectionnonum}[1]
{
	\suppressfloats[t]

	\section*{#1}
	\addcontentsline{toc}{section}{#1}
}


{
	\Huge
	\bfseries
	\centerline{Человек}
}

\tableofcontents

\newpage

\newsection{Падение}

5...\newline

4...\newline

3...\newline

2...\newline

1...\newline

Удар.\newline

Он поразил его органы чувств и сотряс тело, отправив сознание в
небытие.\newline

Когда Василий пришёл в себя, он прислушался к себе, к своему телу. Казалось,
оно было в порядке, хотя и ныло практически везде, где могло.
<<Ничего>> --- подумал Василий --- <<я жив, следовательно, ещё не всё
потеряно>>. Он открыл глаза и оглядел капсулу, в которой находился.
Монитор, отсчитывающий мгновения до удара, был разбит, да и в целом капсула
не походила на то, чем можно дальше пользоваться как ни в чём не бывало.
Вокруг кресла, в котором лежал Василий, были разбросаны осколки мониторов и
обломки частей капсулы. Внутренняя обшивка была исчерчена трещинами.
Похоже, состояние внешней обшивки было более чем плачевным. Об этом же
свидетельствовал еле слышный шум, словно песок легонько скользил по листу
нержавеющей стали.

<<Хорошо, что эта планета пригодна для жизни>> ---
подумал Василий и решил выбираться наружу, чтобы получше понять состояние,
в котором он оказался. Выйти через входной люк не было возможности, так как
его геометрия была нарушена, так что открыть его было нельзя. Но в одном месте
Василий увидел лучик света, пробивавшийся через трещину во внутренней обшивке.
Доломав обшивку до конца, Василий увидел, что внешняя обшивка в этом месте
была разорвана от удара. Покинув капсулу, Василий ступил на влажную землю,
покрытую редкой травой и мхом. Свет солнца едва пробивался через кроны
мощных и высоких хвойных деревьев. Температура была близка к идеальной,
свежий, хоть и неподвижный воздух успокаивал сознание и согревал тело.
Вершины деревьев мягко огибал ветер, хвоя на них и создавала тот самый
лёгкий шум.

<<Если бы это была Земля!>> --- подумал Василий. Но нет, эта была планета,
считавшаяся необитаемой, разумных форм жизни на ней обнаружено не было.
Василий понимал, что только никто не придёт к нему на помощь в ближайшее
десятилетие, он должен подумать о себе сам.

Обойдя капсулу вокруг, Василий сделал вывод, что всё не так уж плохо.
Силовой каркас под внешней обшивкой принял основную часть энергии удара на
себя, во внешней обшивке было только несколько разрывов. Забравшись обратно
в капсулу Василий снл внутреннюю обшивку в тех местах, где были проложены
кабели. Он убедился, что резервная система электроснабжения не повреждена
и включил генератор. Уцелевшая часть капсулы ожила, заработал холодильный
блок с небольшим запасом еды, загорелась единственная уцелевшая лампа.
Капсула потеряла ощущение безжизненного осколка цивилизации, обрела лёгкое
чувство уюта.

Василий, умиротворённый сложившейся атмосферой, почувствовал усталость.
Он сел на кресло и провалился в глубокий сон.

\newsection{Первая разведка}

Василий проснулся. Его тело было отдохнувшим, сознание --- бодрым и готовым
к новым решениям и событиям. Василий вышел из капсулы. Было раннее утро.
Самое время, чтобы пройтись по окрестностям и подумать о выстраивании
своего быта на новом месте. Надо было подумать о поиске еды, воды.
<<Первое время придётся заниматься собирательством>> --- решил Василий.
Запас еды в капсуле он решил отложить на крайний случай.

Василий обошёл окрестности. На юге от капсулы нашлось небольшое озеро.
С помощью очистителя, установленного в капсуле, можно сделать воду из него
пригодной для питья. Василий набрал воды в 3 фляги и понёс их к капсуле.

Оорганизовав запас питьевой воды, Василий продолжил изучение окрестностей.
Ведь он так и не нашёл ничего, что могло бы быть ему пищей. Через некоторое
время он нашёл низкие кусты с ягодами. Они росли под особо крупными деревьями,
кроны которых вростирались ввысь на огромную высоту.

Почти весь день ушёл у Василия на изучение окрестностей, но под самый
вечер нечто странное
увидел он вдали. Словно металл блестел на солнце, закатывающемся за горизонт.
<<Утро вечера мудренее>> --- решил Василий и вернулся в капсулу. Он положил
собранные им ягоды в ящик и погрузился в размышления. Будущее представлялось
ему довольно оптимистичным. У него есть еда и вода, а, насколько было
известно, на этой планете были только растения, так что хищные животные ему не
угрожали. У него есть капсула. Не просто безликая спасательная капсула,
сотни которых он видел на больших кораблях. Это его дом. С такими мыслями
Василий лёг в кресло и уснул.

\newsection{Бункер}

На следующий день, взяв с собой флягу воды и немного ягод, Василий отправился
к тому месту, где он видел блеск металла. Аккуратно подойдя к этому месту,
Василий увидел лист металла, под углом поднимающийся из-под земли.
Обойдя его, Василий увидел, что это было некоторое подобие капсулы, в которой
на планету упал он, только очень старой. Она почти полностью вросла в землю.

Василий подумал: <<Какая бы старая эта штука ни была, вдруг в ней есть что-то
ценное для меня>>. В выступающей над землёй части капсулы был проём, через
который Василий и пролез внутрь. Верхняя часть проёма была увита
переплетающимися ветвями с большим количеством листьев.

Внутри капсулы не было почти ничего, что могло бы представлять интерес.
Уцелел только корпус капсулы, внутри лежали металлические обломки.
Василий понял, что ничего интересного он здесь не найдёт и развернулся к
выходу.

Эта история могла пойти совсем другим путём, но всё случилось так, как
случилось. Разворачиваясь, Василий случайно задел одну из веток. Она
отвалилась от остальных веток и полетела вниз. Острым сучком она вонзилась
прямо в руку Василию. <<Вот ведь пакость!>> --- подумал Василий, искривившись
от боли. <<Надо мне было так неосторожно повернуться!>>

Василий выбрался из капсулы и побежал к дому. Добравшись туда, он кинулся
к стойке с лекарствами. Превозмогая боль, он обработал рану и наложил
бинты. <<Надо же мне было так!>> Эта мысль не отпускала его ни на минуту.

Закончив со своей рукой, Василий вышел на воздух. Прислушался к себе.
Сознание было ясным. Рука давала о себе знать, но в остальном тело Василия
было в порядке. Он успокоил свои мысли и вернулся в дом.
<<Надо немного отдохнуть>> --- решил Василий.

\newsection{Дело серьёзное}

Проснувшись, Василий прислушался к себе. Он ожидал, что дискомфорт в руке
будет меньше, как обычно бывает с травмами по прошествии времени. Но
в этот раз всё было не так. Рука болела. Сильнее и сильнее с каждой минутой.
Не вытерпев этого, Василий вскочил и снял бинт.

Увиденное повергло его в шок. Прямо на том месте, где была рана, торчал
небольшой, но уже одревесневевший стебель с двумя зелёными листочками.
Василия пробил озноб. Мысли путались в голове. Он смотрел на свою
руку и не мог поверить своим глазам.

Очнувшись от оцепенения, Василий попробовал потянуть за стебель. Но тут же
он почувствовал, как стебель тащит за собой его кость. Василий снова впал
в оцепенение. Мысли в голове всё никак не приходили в порядок.
<<Моё тело не в порядке. Не в порядке>>. Эта мысль монотонно звучала
в сознании Василия, заглушая его. Василий тяжело опустился в кресло и
уснул.

Проснувшись, Василий первым делом посмотрел на свою руку. Стебель был на
месте. Более того, он вырос на большую длину и обзавёлся ещё двумя листиками.
Холод снова пробежал по всему телу Василия. Он понимал, что раз стебель
прикрепился к его кости, он не сможет ничего с ним сделать без существенного
вреда для себя. Василий принялся изучать каждый сантиметр вокруг стебля.
Чем больше он изучал, тем больше приходил в ужас. Стебель внутри руки был уже
не один. Он ветвился, прикрепляясь к кости в нескольких местах.
От осознания этого факта эмоциональное состояние Василия оказалось
окончательно перегружено, и он потерял сознание.

\newsection{Он --- это я?}

Несколько раз Василий приходил в сознание, но ненадолго. От сильной головной
боли он снова отправлялся в небытие. Придя в себя через продолжительное
время, Василий прислушался к себе. Сознание шумело, но в целом было ясным.
Но тело. Общие ощущения были те же, что и обычно. Но что-то было не так.
И это что-то было совершенно новым. Оно стучалось в сознание, словно взывая
к нему, требуя к себе внимания. Василий открыл глаза. Стебля не было.
Но чувствовалось, что к кости руки по-прежнему прикреплены ветви,
проходящие между мышечных волокон.

Василий чувствовал сильную жажду.
Он встал с кресла и взял в руку флягу с водой той рукой, из которой раньше
торчал стебель. Как обычно, он думал, что сейчас
он поднесёт флягу ко рту и будет пить. И чувство жажды утихнет.
Но не в этот раз. Как только Василий представил воду во фляге, из его руки
показался стебель. Василий замер, забыв о воде. Стебель тоже замер.
Василий прислушался к себе, к тому новому, что в нём появилось. Коснулся
другой рукой стебля и тот час же почувствовал это прикосновение.
Попробовал ещё раз. И снова почувствовал его. Это чувство прикосновения
было совершенно новым, и оно исходило от стебля. Василий замер, а потом
снова подумал о воде. Потянулся к ней. И стебель, на ходу выращивая листья,
потянулся к фляге и погрузился в воду.

Василий почувствовал, что жажда отпускает его. Он перестал тянуться к воде
своим сознанием, стебель вылез из фляги и исчез в руке.
Василий заглянул внутрь фляги. Она была пуста. Василий поставил её на стол,
сел в кресло и глубоко задумался. Он думал о том, что с ним случилось.
Постепенно его мысли смещались в сторону чувства, всё больше и больше
дававшего о себе знать. Это был голод. Василий сидел, размышляя, идти
ему за ягодами сейчас или чуть позже. Взгляд его перекинулся на одинокую
лампу. И тут снова что-то новое постучалось в его сознание. Он прислушался
к этому чувству. Это была тяга к свету. Она была не той тагой к свету,
что имеет человек, она была большей.

Василий потянулся к ламе своим сознанием, и тут стебель снова показался
из его руки. Он протянулся к лампе, разветвился и покрылся множеством
листьев. Василий почувствовал, что голод отступает, хоть и очень медленно.

Снова в его голове пронёсся вихрь мыслей. И главным был вопрос:
<<Кто я теперь?>>. Теперь сознание Василия не могло точно дать ответ
на этот вопрос. <<Я человек? Да. У меня есть руки и ноги, я могу ходить,
разговаривать, думать в конце концов. Но разве человек может быть с листьями,
может употреблять воду иначе как пить, может фотосинтезировать? Нет.
Я человек? Нет. Я растение? Да. У меня есть листья, стебли, я могу
фотосинтезировать. Но у меня есть руки и ноги, я могу есть и пить, могу
думать. Я растение? Нет. Я не человек. Я не растение. Хотя о чём это я?
Я же человек. Был. Тогда перестал ли я им быть? Что значит это самое
<<быть человеком>>? И что значит быть растением? Странно всё это.
Надо поспать>>.

\newsection{Осознание}

Василий проснулся, прислушался к себе. Его сознание было ясным. Тело
не подавало никаких признаков недомогания. Человеческое тело.
Но появилось и что-то ещё. Второе тело. Сознание ещё не привыкло к нему,
поэтому оно воспринималось особенно непривычно. Оно не имело суставов,
мышц или привычных органов. Оно было облаком. Облаком, способным
перестраиваться, расти, сжиматься, добывать воду и еду.

И оно было Василием. Как и сознание Василия, человеческое тело Василия,
это новое тело теперь тоже было частью Василия. Василий сидел, укладывая
это в своей голове.

Что же, на этом мы оставим Василия на долгих 5 лет.

\newsection{Жизнь}

Василий проснулся, прислушался к себе. Всё было прекрасно. Стебли под ним
пришли в движение, теряя форму кресла и одновременно перемещая Василия в
вертикальное положение. Василий вышел на улицу, огляделся, вдохнул
свежий утренний воздух. Стебли, вьющиеся вокруг его ног, пускали побеги,
высасывающие воду из влажной земли. Ночью был дождь.

Василий потянулся и пошёл к своему любимому дереву, простирающемуся ввысь.
Он обвил ствол стеблями и пошёл вверх, пока не добрался до верхней части
кроны. Там он сел, раскинул листья на солнечном свете и замер.

Этот день был особенным. Долгие годы Василий строил дом. Такой дом,
который позволил бы ему узнать больше о планете, на которой он оказался.
Василий решил, что лучше всего, если его дом сможет перемещаться по планете,
так что он построил большое самоходное устройство, совмещающее в себе
склад, жилище и мастерскую. И в тот день он запланировал отправление.

Чувство начинающегося приключения переполняло Василия. Он спустился с
дерева, вошёл в свой новый дом. Подошёл к пульту управления. Задал маршрут.
Дом пришёл в движение и неторопливо зашагал вперёд, а Василий вылез на крышу,
сел и приготовился к новому этапу своей жизни.

\end{document}
